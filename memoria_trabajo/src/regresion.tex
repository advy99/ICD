\section{Regresión}

En esta sección aplicaremos dos métodos de regresión al conjunto de datos baseball. Los métodos que aplicaremos serán:

\begin{itemize}
	\item Regresión lineal, tanto simple como múltiple.
	\item K-NN.
\end{itemize}

Tras aplicar estos métodos realizaremos una comparación estadistica entre ellos y con un tercer método, el método de regresión M5'.

\subsection{Regresión lineal simple}

El método de regresión lineal simple se trata de encontrar la linea que mejor se ajuste a unos datos de entrenamiento, normalmente utilizando el método de mínimos cuadrados.

En este caso el modelo tendrá la siguiente forma:

\[
Y = \beta_0 + \beta_1 \cdot X + \epsilon
\]

Donde $\beta_0$ y $\beta_1$ son las dos constantes desconocidas que estimaremos con mínimos cuadrados, $\epsilon$ un termino de error, $X$ el valor del predictor e $Y$ el valor real de la variable objetivo.

En este caso, tras obtener los valores de las constantes, tendremos el siguiente modelo:

\[
\hat{y} = \hat{\beta_0} + \hat{\beta_1} \cdot x
\]

Este modelo nos servirá para hacer una predicción del valor de $Y$ que llamaremos $\hat{y}$ gracias a la estimación de las constantes.

En este caso vamos a realizar cinco ajustes con cinco predictores distintos.

\subsubsection{Selección de predictores}

De cara a seleccionar los cinco predictores con los que realizar cada uno de los ajustes he decidido escoger los predictores con mayor correlación con la variable a predecir, el salario. En el código en R adjunto de esta sección está disponible como se ha automatizado esto haciendo uso de la función \texttt{cor} para obtener las correlaciones y obtener las más altas, además del estudio que se realizó en el análisis exploratorio de los datos \ref{fig:matriz_correlaciones_baseball}.

Los predictores escogidos, ordenados de mayor a menor correlación, son:

\begin{itemize}
	\item Runs batted in, con una correlación de 0.6684219 con Salary.
	\item Runs, con una correlación de 0.6429035 con Salary.
	\item Hits, con una correlación de 0.6212390 con Salary.
	\item HomeRuns, con una correlación de 0.5904536 con Salary.
	\item Doubles, con una correlación de 0.5774227 con Salary.
\end{itemize}

\subsubsection{Resultados}

Tras ajustar el modelo con \texttt{lm} obtenemos los siguientes resultados.

El modelo utilizando el predictor Runs batted in consigue un MSE de 848113.1 y un valor de $R^2$ ajustada de 0.4451. En el siguiente gráfico podemos ver la linea de ajuste obtenida:

\begin{figure}[H]
	\centering
	\includesvg[width = 400pt]{baseball/ajuste_lm/ajuste_lm_Runs_batted_in}
	\caption{Ajuste regresión lineal simple en baseball utilizando el predictor Runs\_batted\_in.}
	\label{fig:ajuste_Runs_batted_in}
\end{figure}

El modelo utilizando el predictor Runs consigue un MSE de 899414.2 y un valor de $R^2$ ajustada de 0.4116. En el siguiente gráfico podemos ver la linea de ajuste obtenida:

\begin{figure}[H]
	\centering
	\includesvg[width = 400pt]{baseball/ajuste_lm/ajuste_lm_Runs}
	\caption{Ajuste regresión lineal simple en baseball utilizando el predictor Runs.}
	\label{fig:ajuste_Runs}
\end{figure}

El modelo utilizando el predictor Hits consigue un MSE de 941400.4 y un valor de $R^2$ ajustada de 0.3841. En el siguiente gráfico podemos ver la linea de ajuste obtenida:

\begin{figure}[H]
	\centering
	\includesvg[width = 400pt]{baseball/ajuste_lm/ajuste_lm_Hits}
	\caption{Ajuste regresión lineal simple en baseball utilizando el predictor Hits.}
	\label{fig:ajuste_Hits}
\end{figure}

El modelo utilizando el predictor HomeRuns consigue un MSE de 998587.7 y un valor de $R^2$ ajustada de 0.3467. En el siguiente gráfico podemos ver la linea de ajuste obtenida:

\begin{figure}[H]
	\centering
	\includesvg[width = 400pt]{baseball/ajuste_lm/ajuste_lm_HomeRuns}
	\caption{Ajuste regresión lineal simple en baseball utilizando el predictor HomeRuns.}
	\label{fig:ajuste_HomeRuns}
\end{figure}

El modelo utilizando el predictor Doubles consigue un MSE de 1021919 y un valor de $R^2$ ajustada de 0.3314. En el siguiente gráfico podemos ver la linea de ajuste obtenida:

\begin{figure}[H]
	\centering
	\includesvg[width = 400pt]{baseball/ajuste_lm/ajuste_lm_Doubles}
	\caption{Ajuste regresión lineal simple en baseball utilizando el predictor Doubles.}
	\label{fig:ajuste_Doubles}
\end{figure}

Como era de esperar, a una mayor correlación con la variable de salida, un mejor MSE, y un menor $R^2$, ya que la linea obtenida se aleja de la linea que ajusta de forma perfecta los datos, que tendría un $R^2$ de uno y un MSE de cero.

\begin{table}[H]
	\centering
\begin{tabular}{|c|c|c|c|}
\hline
\textbf{Predictor}      & \textbf{\begin{tabular}[c]{@{}c@{}}Correlación\\ con Salary\end{tabular}} & \textbf{MSE} & \textbf{$R^2$} \\ \hline
\textbf{Runs batted in} & 0.668                                                                     & 848113.1     & 0.4451                        \\ \hline
\textbf{Runs}           & 0.642                                                                     & 899414.2     & 0.4116                        \\ \hline
\textbf{Hits}           & 0.621                                                                     & 941400.4     & 0.3841                        \\ \hline
\textbf{HomeRuns}       & 0.590                                                                     & 998587.7     & 0.3467                        \\ \hline
\textbf{Doubles}        & 0.577                                                                     & 1021919      & 0.3314                        \\ \hline
\end{tabular}
\end{table}


\subsection{Regresión lineal múltiple}

Este modelo se basa en ajustarse a los datos, pero en lugar de utilizar una linea (un solo predictor), utilizar multiples predictores, generando un hiperplano con el que predecir la variable de salida.

En este caso la fórmula tendrá la siguiente forma:

\[
\hat{y} = \hat{\beta_0} + \hat{\beta_1} \cdot x + \cdots + \hat{\beta_n} \cdot x_n
\]

Donde utilizaremos $n$ predictores y por lo tanto tendremos $n + 1$ constantes, que al igual que con regresión lineal simple, estimaremos con el método de mínimos cuadrados.

\subsubsection{Obtención del modelo lineal múltiple básico}

De cara a obtener el modelo lineal múltiple se ha utilizado una metodología decremental, en el que se ha comenzado utilizando todos los predictores posibles para predecir el salario, y en cada pase se ha eliminado un único predictor, el considerado menos relevante por el modelo, de cara a mejorar los resultados y simplificar el modelo completo buscando aumentar su $R^2$. Para decidir que predictor eliminar, en cada modelo generado se ha consultado su información con el comando \texttt{summary} que nos da información sobre si un predictor es relevante a través de un p-value, si dicho p-value es menor a $0.05$ se considera relevante un predictor.

En el código adjunto al final de esta sección está disponible todos los pasos realizados comentando los resultados de cada paso, aquí simplemente realizaré los primeros pasos para explicar la metodología.

Comenzamos con el modelo con todos los predictores y miramos su summary:

\begin{lstlisting}[language=R]
fit_lm_multiple <- lm(Salary ~ ., data = baseball)
summary(fit_lm_multiple)
\end{lstlisting}

Al ejecutar este código vemos que obtenemos un $R^2$ de 0.6865, bastante mayor que el obtenido con regresión simple. También vemos como el predictor Doubles tiene el mayor p-value, con casi un 0.9, por lo que no es relevante para el modelo, así que volvemos a entrenar el modelo pero sin ese predictor:

\begin{lstlisting}[language=R]
fit_lm_multiple <- lm(Salary ~ . - Doubles, data = baseball)
summary(fit_lm_multiple)
\end{lstlisting}

Y con la salida de esta ejecución, volvemos a repetir estos pasos, eliminando en cada paso únicamente un predictor.

Como resultado final se obtiene el siguiente modelo:

\begin{lstlisting}[language=R]
fit_lm_multiple <- lm(Salary ~ . - Doubles - Hits - Batting_average - Walks - `On-base_percentage` - Triples - Arbitration - Errors - Runs - Free_agent, data = baseball)
\end{lstlisting}

Como vemos, hemos eliminado muchos predictores, y nos hemos quedado finalmente con los más importante, obteniendo un $R^2$ de 0.68, mejorando en gran cantidad el modelo de regresión lineal simple y reduciendo el MSE a 480894.9.


\subsubsection{Interacciones y no linealidad}

Tras obtener el modelo lineal múltiple se ha probado a introducir distintas interacciones entre predictores y no linealidad.

Algunas de estas interacciones e introducción de no linealidad ha funcionado, sin embargo otras no, todos las combinaciones que se han probado están especificadas en el código ejecutado con comentarios sobre las pruebas. Cabe destacar que en todo caso se ha tenido en cuenta que si se ha probado a introducir una interacción o no linealidad siempre se ha introducido en el modelo los términos base para que se ajuste de forma correcta.

Se ha probado con varios predictores correlados entre si, como Free agency eligibility y Runs batted in, además de predictores que podría tener sentido que influyeran en el salario, como pueden ser Runs y Hits, funcionando ambas interacciones bastante bien, sin embargo, otras como Batting average con On base percentage que podría parecer que funcionarían no han aportado información al modelo.

De cara a introducir no linealidad se han probado con distintos predictores que no han funcionado en el modelo lineal, esperando que la relación entre el predictor y el salario no fuera lineal, sin embargo solo ha funcionado la introducción de no linealidad con el predictor Doubles con termino cuadrático y cúbico y con el término Errors.

\newpage

\subsubsection{Resultados del modelo final}

Tras realizar el estudio del mejor modelo lineal múltiple e introducir interacciones y no linealidad, el mejor modelo ha obtenido un valor de $R^2$ de 0.75 y un MSE de 367189.1, bastante mejor que con el modelo de regresión lineal simple y el modelo múltiple sin interacciones y no linealidad.

De cara a observar como son las predicciones que realiza, se ha realizado el siguiente gráfico donde podemos ver el salario con respecto al predictor con mayor correlación, Runs batted in, en rojo los valores reales y en azul los valores predichos:

\begin{figure}[H]
	\centering
	\includesvg[width = 500pt]{baseball/predicciones_lm_completo}
	\caption{Predicciones realizadas por el modelo de regresión lineal múltiple con interacciones y no linealidad.}
	\label{fig:predicciones_lm_completo}
\end{figure}

Como vemos, la nube de puntos obtenida se asemeja bastante a la nube de puntos real.

\newpage

\subsection{K-NN}

K-NN para regresión se basa en, para un punto en el espacio donde tenemos que predecir un valor, utilizar como predicción la media de los valores de los K vecinos más cercanos. De cara a buscar los vecinos más cercanos se pueden utilizar distintas medidas de distancia, aunque la más común será la distancia euclidea, la cual utilizaremos en esta sección.

Debido a que se calcularán distancias, es necesario normalizar los datos antes de aplicar el algoritmo. En este caso se ha utilizado la función \texttt{kknn}, que ya incorpora el escalado de datos al crear el modelo. Con respecto a los parámetros a utilizar, se ha mantenido el valor de k por defecto así como el parámetro kernel, el cuál se buscará el óptimo automaticamente, tal y como vimos en clase.

\subsubsection{Modelos utilizados}

De cara a probar K-NN se ha probado con un modelo donde se incluyen todos los predictores así como la fórmula obtenida en el mejor modelo de regresión lineal. En este caso ha funcionado mejor el segundo modelo, en el que se ha obtenido un MSE de 189463.2, reduciendo en la mitad el obtenido por regresión lineal múltiple.

Al igual que con el modelo de regresión lineal múltiple, se ha realizado un gráfico con las predicciones del modelo:

\begin{figure}[H]
	\centering
	\includesvg[width = 500pt]{baseball/predicciones_knn}
	\caption{Predicciones realizadas por el modelo K-NN.}
	\label{fig:predicciones_knn}
\end{figure}

\subsection{Comparación entre algoritmos}

En esta sección compararemos los resultados de regresión lineal y K-NN tras ejecutarlos en 17 conjuntos de datos distintos, incluido el conjunto utilizado en este trabajo.

Tras esta comparación añadiremos un tercer método, el MD5', para comprobar si alguno de estos tres es mejor.

Estas comparaciones las realizaremos utilizando métodos estadísticos, aunque antes tendremos que obtener unos resultados fiables de los algoritmos utilizados para nuestro conjunto de datos, por lo que aplicaremos validación cruzada a los mejores modelos obtenidos.

\subsubsection{Ejecución de los algoritmos con validación cruzada}

Tras lanzar los modelos obtenidos en los apartados anteriores utilizando validación cruzada con los cinco folds dados, se obtienen los siguientes resultados:

% Please add the following required packages to your document preamble:
% \usepackage{graphicx}
\begin{table}[H]
\centering
\resizebox{\textwidth}{!}{%
\begin{tabular}{|c|c|c|}
\hline
\textbf{Método} & \textbf{Train} & \textbf{Test} \\ \hline
\textbf{LM}     & 360383.1       & 434420.4      \\ \hline
\textbf{K-NN}   & 189809         & 558852.9      \\ \hline
\end{tabular}%
}
\caption{MSE obtenido en el conjunto de entrenamiento y test para cada método.}
\end{table}

Como podemos observar, aunque parecía que K-NN se iba a comportar mejor que regresión lineal para este problema, en realidad es que cuenta con un alto sobreajuste, haciendo que aprenda muy bien los datos con los que entrena pero se comporta mal con datos nuevos.

\subsubsection{Comparación entre regresión lineal múltiple y K-NN}

Tras haber obtenido los datos de nuestros modelos para el dataset baseball y haberlos modificado en la tabla con todos los resultados de los distintos conjuntos de datos, realizamos la comparación entre regresión lineal múltiple y K-NN utilizando el test de Wilcoxon.

Para lanzar este test tenemos que obtener la tabla de diferencias normalizadas entre los resultados en regresión lineal multiple y K-NN. Tnemos que tener en cuenta que el test falla si encuentra algún cero en esta tabla, así que lo solucionamos añadiendo un valor poco significativo (0.1) si encuentra un cero.

Tras obtener esta tabla, lanzamos el test, del que obtenemos un p-value de 0.83, por lo que no podemos rechazar la hipotesis nula de que no hay diferencias significativas entre los dos métodos para los conjuntos de datos utilizados.

\subsubsection{Comparación entre múltiples modelos}

Para realizar esta comparación simplemente utilizaremos el test de Friedman para saber si existe alguna diferencia significativa entre una pareja de algoritmos, si el resultado es que si existe una diferencia significativa aplicaremos el test de Holm como post-hoc para saber entre que dos algoritmos se ha encontrado esa diferencia significativa.

Tras lanzar el test de Friedman nos devuelve un p-value de 0.014, con lo que con un nivel de confianza de más del 98\% podemos rechazar la hipotesis nula de que no existen diferencias significativas entre una pareja de algoritmos, por lo que sabemos que hay al menos un algoritmo que es mejor que otro.

Si aplicamos el test de Holm obtendremos la siguiente tabla:

% Please add the following required packages to your document preamble:
% \usepackage{graphicx}
\begin{table}[H]
\centering
\resizebox{\textwidth}{!}{%
\begin{tabular}{|c|c|c|}
\hline
\textbf{}  & \textbf{1} & \textbf{2} \\ \hline
\textbf{2} & 0.58      & -          \\ \hline
\textbf{3} & 0.08      & 0.10      \\ \hline
\end{tabular}%
}
\end{table}

En la que como vemos, existen diferencias significativas a favor de M5', tanto contra regresión lineal con una confianza del 99\% como con K-NN con una confianza del 90\%.

\subsection{Código R}

En esta sección está el código ejecutado en esta sección:

\begin{lstlisting}[language=R]
library(tidyverse)
library(readr)
library(GGally)
library(corrplot)
library(kknn)
library(readr)

set.seed(2)

#
# Funciones
#

calcular_MSE <- function(valores_reales, valores_predecidos) {
	sum((valores_predecidos - valores_reales)^2)/length(valores_reales)
}

calcular_RMSE <- function(valores_reales, valores_predecidos) {
	sqrt(sum((valores_predecidos - valores_reales)^2)/(length(valores_reales)-2))
}

# creamos si es necesario los directorios a usar de salida
directorios_necesarios <- list("out", "out/baseball", "out/baseball/ajuste_lm")

lapply(directorios_necesarios, function(directorio) {if (!dir.exists(directorio)) dir.create(directorio) })


# leemos los datos
baseball <- read.csv("data/baseball/baseball.dat", comment.char = "@", header = FALSE)

# le asignamos los nombres de las columnas
names(baseball) <- c("Batting_average", "On-base_percentage", "Runs", "Hits",
					 "Doubles", "Triples", "HomeRuns", "Runs_batted_in", "Walks",
					 "Strike-Outs", "Stolen_bases", "Errors",
					 "Free_agency_eligibility", "Free_agent",
					 "Arbitration_eligibility", "Arbitration", "Salary")


matriz_correlaciones_baseball <- cor(baseball)
# convertimos la matriz a dataframe
matriz_correlaciones_baseball <- as.data.frame(matriz_correlaciones_baseball)

# nos quedamos con los cinco predictores más correlados
cinco_mas_correladas <- matriz_correlaciones_baseball %>% select(Salary) %>% # nos quedamos solo con la fila de salario
														  top_n(6) %>% # nos quedamos con las 6 mejores (son todas positivas)
														  arrange(desc(Salary)) %>% # las ordenamos por correlacion
														  slice(-c(1)) # eliminamos la primera fila (el propio salario)
nombres_cinco_mas_correladas <- row.names(cinco_mas_correladas)
nombres_cinco_mas_correladas

# saco las formulas del salario dependiendo de cada predictor
formulas_mas_correlados <- lapply(nombres_cinco_mas_correladas, function(predictor) {
	as.formula(paste("Salary ~ ", predictor))
})

# aplico la regresión lineal simple de cada formula
fit_lm_baseball <- lapply(formulas_mas_correlados, function(formula_lm) {
	lm(formula_lm, data = baseball)
})

# miramos el summary de todos los ajustes
lapply(fit_lm_baseball, summary)

# calculamos el MSE
lapply(fit_lm_baseball, function (fit) {
	nombre_predictor <- fit$terms[[3]]
	calcular_MSE(baseball$Salary, predict(fit, baseball$nombre_predictor))
})

# calculamos el RMSE
lapply(fit_lm_baseball, function (fit) {
	nombre_predictor <- fit$terms[[3]]
	calcular_RMSE(baseball$Salary, predict(fit, baseball$nombre_predictor))
})


# guardamos en disco un grafico con el ajuste obtenido
lapply(fit_lm_baseball, function (fit) {
	nombre_predictor <- fit$terms[[3]]
	svg(paste("out/baseball/ajuste_lm/ajuste_lm_", nombre_predictor, ".svg", sep = ""))
	plot(as.formula(paste("Salary ~", nombre_predictor) ), baseball)
	abline(fit, col="blue")
	dev.off()
	par(mfrow=c(1,1))
})

#
# Regresión lineal multiple
#

# aplicamos construcción del modelo top-down

fit_lm_multiple <- lm(Salary ~ ., data = baseball)

summary(fit_lm_multiple)

# conseguimos un adjusted R-sqared de 0.6865

# eliminamos la menos significativas: las que tienen mayor p-value
fit_lm_multiple <- lm(Salary ~ . - Doubles, data = baseball)

summary(fit_lm_multiple)

# conseguimos un adjusted R-sqared de 0.6865

# eliminamos la menos significativas

fit_lm_multiple <- lm(Salary ~ . - Doubles - Hits, data = baseball)

summary(fit_lm_multiple)
# conseguimos un adjusted R-sqared de 0.6878

# eliminamos la menos significativas

fit_lm_multiple <- lm(Salary ~ . - Doubles - Hits - Batting_average, data = baseball)

summary(fit_lm_multiple)
# conseguimos un adjusted R-sqared de 0.6881

# eliminamos la menos significativas
fit_lm_multiple <- lm(Salary ~ . - Doubles - Hits - Batting_average - Walks, data = baseball)

summary(fit_lm_multiple)

# conseguimos un adjusted R-sqared de 0.6883

# eliminamos la menos significativas
fit_lm_multiple <- lm(Salary ~ . - Doubles - Hits - Batting_average - Walks - `On-base_percentage`, data = baseball)

summary(fit_lm_multiple)

# conseguimos un adjusted R-sqared de 0.6886

# eliminamos la menos significativas
fit_lm_multiple <- lm(Salary ~ . - Doubles - Hits - Batting_average - Walks - `On-base_percentage` - Triples, data = baseball)

summary(fit_lm_multiple)

# conseguimos un adjusted R-sqared de 0.6883, aunque hemos bajado, no es nada significativo
# seguimos eliminando predictores

# eliminamos la menos significativas
fit_lm_multiple <- lm(Salary ~ . - Doubles - Hits - Batting_average - Walks - `On-base_percentage` - Triples - Arbitration, data = baseball)

summary(fit_lm_multiple)

# conseguimos un adjusted R-sqared de 0.6874, aunque hemos bajado, no es nada significativo
# seguimos eliminando predictores

# eliminamos la menos significativas
fit_lm_multiple <- lm(Salary ~ . - Doubles - Hits - Batting_average - Walks - `On-base_percentage` - Triples - Arbitration - Errors, data = baseball)

summary(fit_lm_multiple)

# conseguimos un adjusted R-sqared de 0.6865, aunque hemos bajado, no es nada significativo
# seguimos eliminando predictores

# eliminamos la menos significativas
fit_lm_multiple <- lm(Salary ~ . - Doubles - Hits - Batting_average - Walks - `On-base_percentage` - Triples - Arbitration - Errors - Runs, data = baseball)

summary(fit_lm_multiple)


# conseguimos un adjusted R-sqared de 0.685, aunque hemos bajado, no es nada significativo
# seguimos eliminando predictores


# A partir de este punto eliminamos características con un p-value < 0.05, por lo que
# podríamos considerarlas significativas, aun así seguiré para probar un modelo simple
#

# eliminamos la menos significativas
fit_lm_multiple <- lm(Salary ~ . - Doubles - Hits - Batting_average - Walks - `On-base_percentage` - Triples - Arbitration - Errors - Runs - Free_agent, data = baseball)

summary(fit_lm_multiple)


# conseguimos un adjusted R-sqared de 0.68, aunque hemos bajado, no es nada significativo
# seguimos eliminando predictores


# eliminamos la menos significativas
fit_lm_multiple <- lm(Salary ~ . - Doubles - Hits - Batting_average - Walks - `On-base_percentage` - Triples - Arbitration - Errors - Runs - Free_agent - HomeRuns, data = baseball)

summary(fit_lm_multiple)

# conseguimos un adjusted R-sqared de 0.67, en este punto hemos bajado un 1% de golpe,
# y eliminando una variable considerada significativa, por lo que el modelo final será el anterior
# a este


# eliminamos la menos significativas
fit_lm_multiple <- lm(Salary ~ . - Doubles - Hits - Batting_average - Walks - `On-base_percentage` - Triples - Arbitration - Errors - Runs - Free_agent, data = baseball)

summary(fit_lm_multiple)

calcular_MSE(baseball$Salary, predict(fit_lm_multiple, baseball))
calcular_RMSE(baseball$Salary, predict(fit_lm_multiple, baseball))

# como vemos, hemos bajado de 848113.1 de MSE y 923 de RMSE a 480894.9 y 695 respectivamente,
# y de un adjusted R-squared de 0.4451 a uno de 0.6806, mejoras bastante significativas


# probamos interacciones: voy  a probar entre predictores que estén correlados entre si, y
# cada uno de ellos con el salario
# pruebo interaccion HomeRuns y Runs en las que batea
fit_lm_multiple_interacciones <- lm(Salary ~ . - Doubles - Hits - Batting_average - Walks -
									  `On-base_percentage` - Triples - Arbitration - Errors -
									  Runs - Free_agent + Free_agency_eligibility*Runs_batted_in,
									  data = baseball)

summary(fit_lm_multiple_interacciones)


# probamos la interaccion entre Runs y Hits, ya que tiene sentido que estas dos
# tengan relacion con el salario
fit_lm_multiple_interacciones <- lm(Salary ~ . - Doubles - Batting_average - Walks -
										`On-base_percentage` - Triples - Arbitration - Errors -
										 Free_agent + Free_agency_eligibility*Runs_batted_in + Runs*Hits,
										data = baseball)
# esta interacción la ha considerado importante, aunque no ha mejorado mucho el R-squared


summary(fit_lm_multiple_interacciones)

# eliminamos HomeRuns, ya que con esta interacción ha considerado que no es importante
# ya que la información que aporta seguramente la aporte esta nueva interaccion
fit_lm_multiple_interacciones <- lm(Salary ~ . - Doubles - Batting_average - Walks -
										`On-base_percentage` - Triples - Arbitration - Errors -
										 Free_agent - HomeRuns + Free_agency_eligibility*Runs_batted_in + Runs*Hits,
										data = baseball)

summary(fit_lm_multiple_interacciones)

# voy a probar tambien entre batting_average y on-base-percentage, ya que
# creo que puede existir asociación entre el porcentaje de veces que llega a una base
# y la media de veces que una persona batea
fit_lm_multiple_interacciones <- lm(Salary ~ . - Doubles - Walks -
										Triples - Arbitration - Errors -
										Free_agent - HomeRuns + Free_agency_eligibility*Runs_batted_in +
										Runs*Hits + Batting_average*`On-base_percentage`,
										data = baseball)

summary(fit_lm_multiple_interacciones)


# pruebo tambien Runs_batted_in con Arbitration_eligibility
fit_lm_multiple_interacciones <- lm(Salary ~ . - Doubles - Batting_average - Walks -
										`On-base_percentage` - Triples - Arbitration - Errors -
										Free_agent - HomeRuns + Runs*Hits +
										Free_agency_eligibility * Runs_batted_in +
										Arbitration_eligibility * Runs_batted_in,
										data = baseball)

summary(fit_lm_multiple_interacciones)

# tambien ha funcionado bien, r^2 pasa de 0.71 a 0.7367

# calculamos el MSE y RMSE con las interacciones
calcular_MSE(baseball$Salary, predict(fit_lm_multiple_interacciones, baseball))
calcular_RMSE(baseball$Salary, predict(fit_lm_multiple_interacciones, baseball))

# hemos bajado de 480894.9 y 695 MSE y RMSE respectivamente a 391673.8 y 627.7039,
# y de un adjusted R-squared de 0.6806 a uno de 0.7367, mejoras bastante significativas


# pasamos a probar no linealidad

# voy a probar con algunos predictores que no funcionaron de forma lineal
fit_lm_multiple_no_linealidad <- lm(Salary ~ . + Doubles - Batting_average - Walks -
										`On-base_percentage` - Triples - Arbitration - Errors -
										Free_agent - HomeRuns + Runs*Hits +
										Free_agency_eligibility * Runs_batted_in +
										Arbitration_eligibility * Runs_batted_in +
										I(Doubles^2),
										data = baseball)

summary(fit_lm_multiple_no_linealidad)

# tras introducir el termino de no linealidad con doubles este predictor si ha pasado a tener importancia


fit_lm_multiple_no_linealidad <- lm(Salary ~ . + Doubles - Batting_average - Walks -
										`On-base_percentage` - Triples - Arbitration - Errors -
										Free_agent - HomeRuns + Runs*Hits +
										Free_agency_eligibility * Runs_batted_in +
										Arbitration_eligibility * Runs_batted_in +
										I(Doubles^3 + Doubles^2) ,
									data = baseball)

summary(fit_lm_multiple_no_linealidad)

# si en lugar de al cuadrado introducimos la caracteristica al cubo mejoramos
# un poco más el adjusted R-squared hasta llegar al 0.7414

# probamos batting average
fit_lm_multiple_no_linealidad <- lm(Salary ~ . + Doubles + Batting_average - Walks -
										`On-base_percentage` - Triples - Arbitration - Errors -
										Free_agent - HomeRuns + Runs*Hits +
										Free_agency_eligibility * Runs_batted_in +
										Arbitration_eligibility * Runs_batted_in +
										I(Doubles^3 + Doubles^2) +
										I(Batting_average^2),
									data = baseball)

summary(fit_lm_multiple_no_linealidad)

# como vemos no ha funcionado, así que lo eliminamos
# y como doubles ha funcionado, vamos a probar con los triples

fit_lm_multiple_no_linealidad <- lm(Salary ~ . + Doubles - Batting_average - Walks -
										`On-base_percentage` + Triples - Arbitration - Errors -
										Free_agent - HomeRuns + Runs*Hits +
										Free_agency_eligibility * Runs_batted_in +
										Arbitration_eligibility * Runs_batted_in +
										I(Doubles^3 + Doubles^2) +
										I(Triples^2),
									data = baseball)

summary(fit_lm_multiple_no_linealidad)

# como vemos este tampoco ha funcionado

# probamos con Errors
fit_lm_multiple_no_linealidad <- lm(Salary ~ . + Doubles - Batting_average - Walks -
										`On-base_percentage` - Triples - Arbitration + Errors -
										Free_agent - HomeRuns + Runs*Hits +
										Free_agency_eligibility * Runs_batted_in +
										Arbitration_eligibility * Runs_batted_in +
										I(Doubles^3 + Doubles^2) +
										I(Errors^2),
									data = baseball)

summary(fit_lm_multiple_no_linealidad)

# si lo considera significativo, subiendo el adjusted R-squared a 0.7499

# probamos con Errors al cubo
fit_lm_multiple_no_linealidad <- lm(Salary ~ . + Doubles - Batting_average - Walks -
										`On-base_percentage` - Triples - Arbitration + Errors -
										Free_agent - HomeRuns + Runs*Hits +
										Free_agency_eligibility * Runs_batted_in +
										Arbitration_eligibility * Runs_batted_in +
										I(Doubles^3 + Doubles^2) +
										I(Errors^2 + Errors^3),
									data = baseball)

summary(fit_lm_multiple_no_linealidad)

# mejora un poco, pero nada significativo

# este será nuestro modelo final
calcular_MSE(baseball$Salary, predict(fit_lm_multiple_no_linealidad, baseball))
calcular_RMSE(baseball$Salary, predict(fit_lm_multiple_no_linealidad, baseball))

# hemos bajado de 391673.8 y 627.7039 MSE y RMSE respectivamente a 367189.1 y 607.7675,
# y de un adjusted R-squared de 0.7367 a uno de 0.7501, mejoras bastante significativas


# predecimos las del mejor modelo
predicciones_train <- predict(fit_lm_multiple_no_linealidad, baseball)

ggplot(baseball %>% mutate(Salary_predicho = predicciones_train) %>% pivot_longer(Salary:Salary_predicho), aes(x = Runs_batted_in, y = value, color = name)) +
	geom_point() +
	labs(title = "Predicciones utilizando el modelo lineal\n múltiple con interacciones y no linealidad",
		 subtitle = "Utilizando característica con mayor correlación con el salario.") +
	ylab("Salary")
ggsave("out/baseball/predicciones_lm_completo.svg", device = svg, width = 1920, height = 1080, units = "px", dpi = 150)


#
# KNN
#

# por defecto kernel = optimal y scale = TRUE
fit_knn <- kknn(Salary ~ ., baseball, baseball)

calcular_MSE(baseball$Salary, fit_knn$fitted.values)
calcular_RMSE(baseball$Salary, fit_knn$fitted.values)

fit_knn_mejor_lm <- fit_knn <- kknn(Salary ~ . + Doubles - Batting_average - Walks -
										`On-base_percentage` - Triples - Arbitration + Errors -
										Free_agent - HomeRuns + Runs*Hits +
										Free_agency_eligibility * Runs_batted_in +
										Arbitration_eligibility * Runs_batted_in +
										I(Doubles^3 + Doubles^2) +
										I(Errors^2 + Errors^3),
									baseball, baseball)

calcular_MSE(baseball$Salary, fit_knn_mejor_lm$fitted.values)
calcular_RMSE(baseball$Salary, fit_knn_mejor_lm$fitted.values)

# como vemos la opción con la formula de lm es algo mejor, pero tampoco mucho


ggplot(baseball %>% mutate(Salary_predicho = fit_knn_mejor_lm$fitted.values) %>% pivot_longer(Salary:Salary_predicho), aes(x = Runs_batted_in, y = value, color = name)) +
	geom_point() +
	labs(title = "Predicciones utilizando KNN",
		 subtitle = "Utilizando característica con mayor correlación con el salario.") +
	ylab("Salary")
ggsave("out/baseball/predicciones_knn.svg", device = svg, width = 1920, height = 1080, units = "px", dpi = 150)



nombre <- "data/baseball/baseball"

run_lm_fold <- function(i, x, tt = "test") {
	file <- paste(x, "-5-", i, "tra.dat", sep="")
	x_tra <- read.csv(file, comment.char="@", header=FALSE)
	file <- paste(x, "-5-", i, "tst.dat", sep="")
	x_tst <- read.csv(file, comment.char="@", header=FALSE)
	names(x_tra) <- c("Batting_average", "On-base_percentage", "Runs", "Hits",
						 "Doubles", "Triples", "HomeRuns", "Runs_batted_in", "Walks",
						 "Strike-Outs", "Stolen_bases", "Errors",
						 "Free_agency_eligibility", "Free_agent",
						 "Arbitration_eligibility", "Arbitration", "Salary")

	names(x_tst) <- c("Batting_average", "On-base_percentage", "Runs", "Hits",
					  "Doubles", "Triples", "HomeRuns", "Runs_batted_in", "Walks",
					  "Strike-Outs", "Stolen_bases", "Errors",
					  "Free_agency_eligibility", "Free_agent",
					  "Arbitration_eligibility", "Arbitration", "Salary")

	if (tt == "train") {
		test <- x_tra
	}
	else {
		test <- x_tst
	}

	fitMulti=lm(Salary ~ . + Doubles - Batting_average - Walks -
					`On-base_percentage` - Triples - Arbitration + Errors -
					Free_agent - HomeRuns + Runs*Hits +
					Free_agency_eligibility * Runs_batted_in +
					Arbitration_eligibility * Runs_batted_in +
					I(Doubles^3 + Doubles^2) +
					I(Errors^2 + Errors^3), data = x_tra)

	yprime=predict(fitMulti,test)
	sum(abs(test$Salary-yprime)^2)/length(yprime) ##MSE
}

lmMSEtrain<-mean(sapply(1:5,run_lm_fold,nombre,"train"))
lmMSEtest<-mean(sapply(1:5,run_lm_fold,nombre,"test"))



run_knn_fold <- function(i, x, tt = "test") {
	file <- paste(x, "-5-", i, "tra.dat", sep="")
	x_tra <- read.csv(file, comment.char="@", header=FALSE)
	file <- paste(x, "-5-", i, "tst.dat", sep="")
	x_tst <- read.csv(file, comment.char="@", header=FALSE)
	In <- length(names(x_tra)) - 1
	names(x_tra) <- c("Batting_average", "On-base_percentage", "Runs", "Hits",
					  "Doubles", "Triples", "HomeRuns", "Runs_batted_in", "Walks",
					  "Strike-Outs", "Stolen_bases", "Errors",
					  "Free_agency_eligibility", "Free_agent",
					  "Arbitration_eligibility", "Arbitration", "Salary")

	names(x_tst) <- c("Batting_average", "On-base_percentage", "Runs", "Hits",
					  "Doubles", "Triples", "HomeRuns", "Runs_batted_in", "Walks",
					  "Strike-Outs", "Stolen_bases", "Errors",
					  "Free_agency_eligibility", "Free_agent",
					  "Arbitration_eligibility", "Arbitration", "Salary")

	if (tt == "train") {
		test <- x_tra
	}
	else {
		test <- x_tst
	}

	fitMulti=kknn(Salary ~ . + Doubles - Batting_average - Walks -
					`On-base_percentage` - Triples - Arbitration + Errors -
					Free_agent - HomeRuns + Runs*Hits +
					Free_agency_eligibility * Runs_batted_in +
					Arbitration_eligibility * Runs_batted_in +
					I(Doubles^3 + Doubles^2) +
					I(Errors^2 + Errors^3),x_tra, test)

	yprime=fitMulti$fitted.values
	sum(abs(test$Salary - yprime)^2)/length(yprime) ##MSE
}

knnMSEtrain <- mean(sapply(1:5,run_knn_fold,nombre,"train"))
knnMSEtest <- mean(sapply(1:5,run_knn_fold,nombre,"test"))


lmMSEtrain
lmMSEtest

knnMSEtrain
knnMSEtest

tabla_resultados_test <- read.csv("data/regr_test_alumnos.csv")
tabla_resultados_test

tabla_resultados_test[5, 2] <- lmMSEtest
tabla_resultados_test[5, 3] <- knnMSEtest
tabla_resultados_test

tabla_resultados_train <- read.csv("data/regr_train_alumnos.csv")
tabla_resultados_train

tabla_resultados_train[5, 2] <- lmMSEtrain
tabla_resultados_train[5, 3] <- knnMSEtrain
tabla_resultados_train

#
# Comparativa LM y KNN con los resultados obtenidos
#

tablatst <- cbind(tabla_resultados_test[,2:dim(tabla_resultados_test)[2]])
colnames(tablatst) <- names(tabla_resultados_test)[2:dim(tabla_resultados_test)[2]]
rownames(tablatst) <- tabla_resultados_test[,1]

# sacamos las diferencias usando la tabla de test
difs <- (tablatst[,1] - tablatst[,2]) / tablatst[,1]
wilc_1_2 <- cbind(ifelse (difs<0, abs(difs)+0.1, 0+0.1), ifelse (difs>0, abs(difs)+0.1, 0+0.1))
colnames(wilc_1_2) <- c(colnames(tablatst)[1], colnames(tablatst)[2])
head(wilc_1_2)

LMvsKNNtst <- wilcox.test(wilc_1_2[,1], wilc_1_2[,2], alternative = "two.sided", paired=TRUE)
Rmas <- LMvsKNNtst$statistic
pvalue <- LMvsKNNtst$p.value
LMvsKNNtst <- wilcox.test(wilc_1_2[,2], wilc_1_2[,1], alternative = "two.sided", paired=TRUE)
Rmenos <- LMvsKNNtst$statistic
Rmas
Rmenos
pvalue

# como el p-value es tan alto, no podemos decir que haya diferencias significativas
# entre LM y KNN para los conjuntos de datos que hemos usado


#
# Comparativa de los tres algoritmos usando Friedman y Holms
#

test_friedman <- friedman.test(as.matrix(tablatst))
test_friedman

# el test de friedman nos dice que con un nivel de confianza de más del 98%,
# si hay diferencias significativas entre alguno de los tres metodos

tam <- dim(tablatst)
groups <- rep(1:tam[2], each=tam[1])
pairwise.wilcox.test(as.matrix(tablatst), groups, p.adjust = "holm", paired = TRUE)

# con este resultado podemos concluir que hay diferencias significativas entre
# el tercer y el primer algoritmo (a favor del tercer algoritmo),
# pero no entre el primero y el segundo y el segundo y el terceo (aunque con este último
# si que las hay si el nivel de confianza es del 90% y no del 5%)
# Por lo tanto, el tercer algoritmo (M5) es mejor que el primero, pero con el
# segundo no tenemos tan claro una mejora

\end{lstlisting}
