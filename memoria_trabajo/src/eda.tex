\section{Análisis exploratorio de datos}

En este primer apartado realizaremos el análisis de los datos. Este análisis inicial nos servirá para visualizar como son los datos, si existen relaciones entre las distintas características que los conforman, así como extraer información de cada característica como pueden ser sus valores de interés, si cuenta con valores perdidos o anomalías así como más información. Además, también nos servirá para realizar hipótesis sobre como se comportarán los distintos predictores a la hora de generar un modelo, si algunos modelos no funcionarán debido a como están distribuidos los datos, entre otras.

\subsection{Conjunto de datos para regresión: baseball}

\subsubsection{Descripción del conjunto de datos}

Este conjunto de datos contiene información sobre distintas estadísticas de jugadores de béisbol, como el número de veces que batean, el número de partidos que juegan, los errores, entre otras estadísticas, así como su salario, que será el valor de salida que intentamos predecir.

Tras esta descripción del conjunto de datos, vamos a pasar a obtener distintas medidas de interés sobre los datos.


\subsubsection{Medidas de interes}

Lo primero a consultar de nuestro conjunto de datos es las dimensiones del mismo. Tras hacer uso del comando \texttt{dim}, vemos que el conjunto cuenta con 337 filas y 17 características (contando con la variable objetivo).

Tras esto pasamos a consultar el tipo de las características de cara a comprobar que se han leído de forma correcta. El comando \texttt{str} nos devolverá esta información, y vemos como todas son de tipo entero, a excepción del porcentaje de bateo y el porcentaje que un jugador llega a una base, que son de tipo numérico al contar con decimales.

Lo siguiente que comprobaremos, antes de consultar los propios valores de las variables será comprobar si existen valores perdidos. Para esto utilizaremos el comando \texttt{is.na} junto con el comando \texttt{any} para saber si se encuentra algún valor perdido. De forma gráfica también podemos utilizar el gráfico \texttt{missmap} del paquete Amelia:

\begin{figure}[H]
	\centering
	\includesvg[width = 400pt]{baseball/missmap_baseball}
	\caption{Mapa de valores perdidos en el conjunto de datos baseball.}
	\label{fig:missmap_iris}
\end{figure}

\subsubsection{Visualización gráfica del conjunto de datos}

\subsubsection{Conclusiones obtenidas a partir del análisis}



\subsection{Conjunto de datos para clasificación: iris}

\subsubsection{Descripción del conjunto de datos}

\subsubsection{Medidas de interes}

\subsubsection{Visualización gráfica del conjunto de datos}

\subsubsection{Conclusiones obtenidas a partir del análisis}
