\section{Clasificación}

En esta sección aplicaremos tres métodos de clasificación al conjunto de datos iris, así como realizar una comparativa entre estos tres métodos. Los métodos serán los siguientes:

\begin{itemize}
	\item K-NN
	\item LDA
	\item QDA
\end{itemize}

\subsection{K-NN}

K-NN para clasificación se basa en, para un punto en el espacio donde tenemos que predecir un valor, utilizar como predicción la etiqueta mayoritaria de los K vecinos más cercanos. De cara a buscar los vecinos más cercanos se pueden utilizar distintas medidas de distancia, aunque la más común será la distancia euclidea, la cual utilizaremos en esta sección.

Debido a que se calcularán distancias es necesario normalizar los datos antes de aplicar el algoritmo, esto es importante ya que si no las variables cuyos valores sean mayores tendrán más peso a la hora de escoger los vecinos, de ahí la importancia de esta normalización. En nuestro caso se ha utilizado la normalización por Z-score, dejando la media de todos predictores en cero y la desviación estándar en uno.

De cara al entrenamiento se ha separado en un conjunto de entrenamiento y otro de test, siendo el conjunto de test un 20\% de las observaciones obtenidas de forma aleatoria con una distribución uniforme sobre el conjunto de datos. Para realizar el entrenamiento se ha utilizado la función \texttt{train} del paqete \texttt{caret}, ya que nos permite hacer una búsqueda de hiperparámetros y aplicar validación cruzada de una forma muy sencilla.

\subsubsection{Elección del parámetro k}

De cara a obtener el mejor valor posible del parámetro k, el número de vecinos más cercanos que se tendrá en cuenta para realizar la estimación, se ha utilizado el parámetro \texttt{tuneGrid} de \texttt{train}. Se han probado valores de k entre 1 y 20, con especial mención a que he decidido utilizar valores pares para el valor de k ya que caret resuelve los empates que se pueden generar, y en nuestro problema, al tener tres clases, incluso con algunos valores impares se podría dar un empate que con un k de valor par no.

Para el entrenamiento del modelo se ha utilizado el conjunto de entrenamiento con validación cruzada de diez particiones, y además se ha calculado la precisión en test para los distintos valores de k para comparar el comportamiento entre entrenamiento y test y así poder comentar sus principales diferencias. Se ha obtenido el siguiente resultado:

\begin{figure}[H]
	\centering
	\includesvg[width = 500pt]{iris/avance_accuracy_iris}
	\caption{Precisión en el conjunto de iris con K-NN para distintos valores de k.}
	\label{fig:avance_accuracy_iris}
\end{figure}

Como podemos ver, este conjunto de datos es muy simple y se obtienen muy buenos resultados, llegando a tener valores de precisión muy alto, siempre por encima el 95\% de acierto en test. Vemos como en entrenamiento con validación cruzada se obtienen los mejores resultados con un k de cinco o seis, y ya que a la hora de escoger el valor de k solo disponemos de la información en el conjunto de entrenamiento, estos son los mejores valores y por lo tanto el valor final que se escogerá de k.

\subsubsection{Resultados}

Tras escoger el mejor valor se han realizado las predicciones en el conjunto de test para ver como es el comportamiento del modelo para datos que no ha observado, obteniendo la siguiente matriz de confusión:

\begin{figure}[H]
	\centering
	\includesvg[width = 500pt]{iris/matriz_confusion_knn}
	\caption{Matriz de confusión del conjunto de test tras aplicar K-NN.}
	\label{fig:matriz_confusion_knn}
\end{figure}

Como vemos, no se ha equivocado en ningún valor del conjunto de test, por lo que el modelo no ha sobreaprendido. También se ha realizado un gráfico para observar como se ha comportado utilizando al clasificar, mostrando las dimensiones del pétalo ya que estos predictores eran los que menor solapamiento mostraban entre las clases como vimos en el análisis exploratorio de datos:

\begin{figure}[H]
	\centering
	\includesvg[width = 500pt]{iris/predicciones_knn_iris}
	\label{fig:predicciones_knn}
\end{figure}

Como vemos, el obtener un 100\% de precisión también se puede explicar con que se ha tenido bastante suerte al escoger el conjunto de test, y no hay muchos datos por la zona donde se solapan las dos clases problemáticas, Versicolor y Virginica.

\subsection{LDA}

\subsubsection{Comprobación de las asunciones de LDA}

\subsubsection{Resultados}

\subsection{QDA}

\subsubsection{Comprobación de las asunciones}

\subsubsection{Resultados}


\subsection{Comparación entre múltiples modelos}


\subsection{Código R}

En esta sección está el código ejecutado en esta sección:


\begin{lstlisting}[language=R]
library(tidyverse)
library(readr)
library(GGally)
library(corrplot)
library(class)
library(caret)
library(ggplot2)
library(scales)
library(philentropy)
library(MASS)
library(car)


set.seed(0)


calcular_accuracy <- function(valores_reales, valores_predichos) {
	sum(valores_reales == valores_predichos) / length(valores_reales)
}

plot_matriz_confusion <- function(matriz, titulo = ""){
	subtitulo <- paste("Accuracy: ", round(matriz$overall[1] * 100, 2), "% \t",
					 "Kappa: ", round(matriz$overall[2] * 100, 2), "%")

	grafico <- ggplot(data = as.data.frame(matriz$table) ,
				   aes(x = Reference, y = Prediction)) +
				   labs(title = titulo, subtitle = subtitulo) +
				   xlab("Valor real") +
				   ylab("Valor predicho") +
				   geom_tile(aes(fill = log(Freq)), colour = "white") +
				   scale_fill_gradient(low = "white", high = "steelblue") +
				   geom_text(aes(x = Reference, y = Prediction, label = Freq)) +
				   theme(legend.position = "none")

	grafico
}

# leemos el fichero de datos
iris <- read.csv("data/iris/iris.dat", comment.char = "@", header = FALSE)

names(iris) <- c("SepalLength", "SepalWidth", "PetalLength", "PetalWidth", "Class")

str(iris)
iris$Class <- as.factor(iris$Class)
str(iris)

# aunque las medidas de iris están en las mismas unidades, y son relativamente
# similares, aun así vamos a normalizarlas ya que no sabemos a priori si alguna
# tendrá más peso al clasificar con knn, de forma que todas tengan la misma importancia

iris <- iris %>% mutate_if(is.numeric, scale, center = TRUE, scale = TRUE)

summary(iris)

# separamos en training y test
PORCENTAJE_TEST = 0.2
muestras_train <- sample(nrow(iris), nrow(iris) * (1 - PORCENTAJE_TEST))

# aprovechamos y le quitamos la clase
iris_train <- iris[muestras_train, -5]
iris_test <- iris[-muestras_train, -5]

nrow(iris_train)
nrow(iris_test)

iris_train_etiquetas <- iris[muestras_train, 5]
iris_test_etiquetas <- iris[-muestras_train, 5]


# no hace falta escalar, ya hemos escalado antes
# buscamos la mejor k entre 1 y 20
modelo_knn <- train(iris_train, iris_train_etiquetas, method = "knn",
					metric="Accuracy", tuneGrid = data.frame(.k=1:20),
					trControl = trainControl(method = "cv", number = 10))

# miramos el resultado, que nos dirá la mejor k
modelo_knn

# para ver el avance en test, entrenamos un knn con las distintas k y calculamos su accuracy en test
resultados_test <- lapply(1:20, function(x) {
	# entreno un modelo con el valor de k = x con validación cruzada y predigo con ese modelo
	modelo_ac_test <- train(iris_train, iris_train_etiquetas, method = "knn",
						metric="Accuracy", tuneGrid = data.frame(.k=x),
						trControl = trainControl(method = "cv", number = 10))
	predict(modelo_ac_test, iris_test)
})
resultados_test <- sapply(resultados_test, function(x) {calcular_accuracy(iris_test_etiquetas, x)})
resultados_test

avance_accuracy <- data.frame(k = 1:20, Accuracy_test = resultados_test, Accuracy_train = modelo_knn$results$Accuracy)

# mostramos avance de accuracy para train y test y guardamos los ficheros
ggplot(avance_accuracy %>% pivot_longer(c(2, 3)), aes(x = k, y = value, color = name)) +
	geom_line() +
	geom_point() +
	ggtitle("Accuracy del modelo en función del valor de K escogido") +
	xlab("Valor de K") +
	ylab("Accuracy")

ggsave("out/iris/avance_accuracy_iris.svg", device = svg, width = 1920, height = 1080, units = "px", dpi = 150)


# predecimos con los datos de test
predicciones_test <- predict(modelo_knn, newdata = iris_test)

# miramos accuracy y kappa con la matriz de confusión
matriz_confusion <- confusionMatrix(predicciones_test, iris_test_etiquetas)
matriz_confusion

plot_matriz_confusion(matriz_confusion, "Modelo K-NN")

ggsave("out/iris/matriz_confusion_knn.svg", device = svg, width = 1920, height = 1080, units = "px", dpi = 150)

iris_plot <- iris
# le cambiamos las etiquetas a las predichas
iris_plot[-muestras_train, "Class"] <- predicciones_test
# marco las que he usado como test
iris_plot["test"] <- 0
iris_plot[-muestras_train, "test"] <- 1
iris_plot$test <- as.factor(iris_plot$test)
str(iris_plot)

ggplot(iris_plot, aes(x = PetalLength, y = PetalWidth, color = Class, shape = test)) +
	geom_point(size = 5) +
	labs(title = "Predicciones utilizando KNN")
ggsave("out/iris/predicciones_knn.svg", device = svg, width = 1920, height = 1080, units = "px", dpi = 150)



#
# Ejercicio 2: LDA
#


# primero tenemos que comprobar las asunciones previas para poder aplicar LDA

# 1. Las observaciones se han obtenido de forma aleatoria (tendremos que suponer que así ha sido)


# 2. Cada clase sigue una distribución normal

# comprobamos por separado en cada clase
apply(iris %>% filter(as.integer(Class) == 1) %>% select_if(is.numeric), 2, shapiro.test)
apply(iris %>% filter(as.integer(Class) == 2) %>% select_if(is.numeric), 2, shapiro.test)
apply(iris %>% filter(as.integer(Class) == 3) %>% select_if(is.numeric), 2, shapiro.test)

# como vemos, a excepción del último predictor, no podemos rechazar que siga una distribución normal
# para las tres etiquetas que tenemos

# 3. Misma matriz de covarianza


bartlett.test(SepalLength ~ Class, iris) # rechazamos hipotesis nula, no sigue la misma covarianza
bartlett.test(SepalWidth ~ Class, iris)
bartlett.test(PetalLength ~ Class, iris) # rechazamos hipotesis nula, no sigue la misma covarianza

# OJO de este no nos podemos fiar, esta variable no sigue una normal
# como vimos en el apartado 2
bartlett.test(PetalWidth ~ Class, iris)

# con LeveneTest se rechaza también H0, así que tampoco tiene la misma covarianza
# (con respecto al as.numeric, me da error si no lo uso, aunque ya es numerico, no se por que)
leveneTest(as.numeric(PetalWidth) ~ Class, iris)

# como vemos, debido a que PetalWidth no sigue una normal dentro de cada clase, y
# que no tenemos la misma matriz de covarianza, no deberíamos aplicar LDA, o por
# lo menos no podemos esperar que nos de muy buenos resultados
# Lo vamos a lanzar aun así.

lda_fit <- train(iris_train, iris_train_etiquetas,
				method = "lda",
				trControl = trainControl(method = "cv", number = 10))

lda_fit

# predecimos con los datos de test
predicciones_test_lda <- predict(lda_fit, newdata = iris_test)

# miramos accuracy y kappa con la matriz de confusión
matriz_confusion <- confusionMatrix(predicciones_test_lda, iris_test_etiquetas)
matriz_confusion

plot_matriz_confusion(matriz_confusion, "Modelo LDA")

ggsave("out/iris/matriz_confusion_lda.svg", device = svg, width = 1920, height = 1080, units = "px", dpi = 150)


iris_plot <- iris
# le cambiamos las etiquetas a las predichas
iris_plot[-muestras_train, "Class"] <- predicciones_test_lda
# marco las que he usado como test
iris_plot["test"] <- 0
iris_plot[-muestras_train, "test"] <- 1
iris_plot$test <- as.factor(iris_plot$test)
str(iris_plot)

ggplot(iris_plot, aes(x = PetalLength, y = PetalWidth, color = Class, shape = test)) +
	geom_point(size = 5) +
	labs(title = "Predicciones utilizando LDA")
ggsave("out/iris/predicciones_lda.svg", device = svg, width = 1920, height = 1080, units = "px", dpi = 150)


#
# QDA
#

# para las asunciones, son las mismas de LDA, pero sin la de la matriz de covarianza
# así que ya lo hemos hecho

qda_fit <- train(iris_train, iris_train_etiquetas,
				 method = "qda",
				 trControl = trainControl(method = "cv", number = 10))

qda_fit


# predecimos con los datos de test
predicciones_test_qda <- predict(qda_fit, newdata = iris_test)

# miramos accuracy y kappa con la matriz de confusión
matriz_confusion <- confusionMatrix(predicciones_test_qda, iris_test_etiquetas)
matriz_confusion

plot_matriz_confusion(matriz_confusion, "Modelo QDA")

ggsave("out/iris/matriz_confusion_qda.svg", device = svg, width = 1920, height = 1080, units = "px", dpi = 150)

iris_plot <- iris
# le cambiamos las etiquetas a las predichas
iris_plot[-muestras_train, "Class"] <- predicciones_test_qda
# marco las que he usado como test
iris_plot["test"] <- 0
iris_plot[-muestras_train, "test"] <- 1
iris_plot$test <- as.factor(iris_plot$test)
str(iris_plot)

ggplot(iris_plot, aes(x = PetalLength, y = PetalWidth, color = Class, shape = test)) +
	geom_point(size = 5) +
	labs(title = "Predicciones utilizando QDA")
ggsave("out/iris/predicciones_qda.svg", device = svg, width = 1920, height = 1080, units = "px", dpi = 150)




#
# comparacion entre los tres modelos
#
nombre <- "data/iris/iris"

run_method_fold <- function(i, x, metodo) {
	# leemos los datos
	file <- paste(x, "-10-", i, "tra.dat", sep="")
	x_tra <- read.csv(file, comment.char="@", header=FALSE)
	file <- paste(x, "-10-", i, "tst.dat", sep="")
	x_tst <- read.csv(file, comment.char="@", header=FALSE)

	names(x_tra) <- c("SepalLength", "SepalWidth", "PetalLength", "PetalWidth", "Class")

	names(x_tst) <- c("SepalLength", "SepalWidth", "PetalLength", "PetalWidth", "Class")

	# pasamos a factor la clas
	x_tra$Class <- as.factor(x_tra$Class)

	x_tst$Class <- as.factor(x_tst$Class)

	# separamos los datos de las etiquetas
	x_tra_labels <- x_tra[, 5]
	x_tra <- x_tra[, -5]

	x_tst_labels <- x_tst[, 5]
	x_tst <- x_tst[, -5]

	# si es el knn, usamos el mejor k para entrenar
	if (metodo == "knn") {
		modelo <- train(x_tra, x_tra_labels,
					   method = metodo,
					   tuneGrid = expand.grid(k = modelo_knn$bestTune$k))
	} else {
		modelo <- train(x_tra, x_tra_labels,
					   method = metodo)
	}

	# predecimos y devolvemos el accuracy
	yprime_train <- predict(modelo,x_tra)
	yprime_test <- predict(modelo,x_tst)
	# calculamos accuracy
	accuracy_train <- calcular_accuracy(x_tra_labels, yprime_train)
	accuracy_test <- calcular_accuracy(x_tst_labels, yprime_test)

	list(train = accuracy_train, test = accuracy_test)

}

# metodos a usar
metodos <- list(knn = "knn", lda = "lda", qda = "qda")
# probe a tener una lista de parametros y pasarsela para cada modelo, sabiendo
# el nombre del modelo, pero a la función train de caret no le gusta y no he encontrado
# forma de pasarles argumentos dependiendo del modelo
# parametros a usar para cada metodo
#parametros_modelos <- list(knn = list(k = modelo_knn$bestTune$k),  # para knn escogemos la mejor k obtenida
#						   lda = list(),
#						   qda = list())

resultados <- lapply(metodos, function(x) {
	# obtenemos los resultados de un modelo
	resultados_metodo <- sapply(1:10, run_method_fold, nombre, x)
	# sacamos la media de train y test y ese es el resultado
	media_train <- mean(unlist(resultados_metodo["train",]))
	media_test <- mean(unlist(resultados_metodo["test",]))
	list(train = media_train, test = media_test)
})

resultados

# leemos y cambiamos los valores obtenidos en test
tabla_resultados_test <- read.csv("data/clasif_test_alumnos.csv")
tabla_resultados_test

tabla_resultados_test[9, 2] <- resultados$knn$test
tabla_resultados_test[9, 3] <- resultados$lda$test
tabla_resultados_test[9, 4] <- resultados$qda$test

tabla_resultados_test


# lo mismo pero en train
tabla_resultados_train <- read.csv("data/clasif_train_alumnos.csv")
tabla_resultados_train

tabla_resultados_train[9, 2] <- resultados$knn$train
tabla_resultados_train[9, 3] <- resultados$lda$train
tabla_resultados_train[9, 4] <- resultados$qda$train


tabla_resultados_train

# comparamos los tres algoritmos

# sacamos la tabla de resultados
tablatst <- cbind(tabla_resultados_test[,2:dim(tabla_resultados_test)[2]])
colnames(tablatst) <- names(tabla_resultados_test)[2:dim(tabla_resultados_test)[2]]
rownames(tablatst) <- tabla_resultados_test[,1]

# se la pasamos al test de friedman
test_friedman <- friedman.test(as.matrix(tablatst))
test_friedman

# como el test nos devuelve un pvalue muy alto, del 0.70, no rechazamos la hipotesis
# nula, luego NO hay diferencias significativas entre ningún par de algoritmos
# con los conjuntos de datos que hemos utilizado
\end{lstlisting}
